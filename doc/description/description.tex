\documentclass[12pt,a4paper]{article}
\usepackage[utf8]{inputenc}
\usepackage{geometry}
\usepackage[english]{babel}
\usepackage[T1]{fontenc}
\usepackage{amsmath}
\usepackage{amsfonts}
\usepackage{amssymb}
%\usepackage{mathrsfs}
\usepackage{listings}
\usepackage{listingsutf8}
\usepackage{graphicx}
\usepackage{float}
\usepackage{hyperref}
\usepackage{tikz}
%\usepackage{tikz-qtree}
\usetikzlibrary{automata}
\usetikzlibrary{positioning}
\usetikzlibrary{shapes}

%\geometry{hmargin=2.0cm,vmargin=2.0cm}
\geometry{left=3cm, right=3cm, top=3cm, bottom=3cm}

\author{Lena DAVID}
\title{Matryoshkryption}
\date{\today}


\begin{document}
\maketitle
\vspace{4em}

\section*{\textit{Abstract}}
    Steganography can be defined as the process of concealing a piece of
    information or data inside another one, so that a third party cannot become
    aware of the presence of the embedded data. Such process can be useful in
    manifold situations, as the fact that it has been used since Ancient Greece
    shows.\\
    \indent Nowadays though, the techniques commonly used are far different
    from these of Ancient times, and steganography mostly applies to digital
    content. In particular, something quite interesting consists in combining
    it with cryptography, as this provides a means to benefit from the best of
    both worlds: assuming you use cryptography correctly, the data you conceal
    is confidential and thus safe from unwanted eyes, and what is more, the
    very fact that there is encrypted data in the cover file used is
    undetectable.\\
    \indent The present challenge aims to provide whomever may tackle it with some
    insights on classical steganographic techniques for different kinds of
    files, as well as on how combining steganography with cryptography can be
    implemented practically.\\

    The following section describes two steganographic techniques that are at
    the core of the realization of the challenge. The last section describes
    in more detail what common methods are used in each step, and how each step
    leads to the next one.\\

    Please note that this challenge was designed with pedagogical purposes and
    should thus not be considered as an illustration of a realistic practical
    implementation.

\section*{Matryoshkryption -- Core principles}
    % angecryption
    % hybrid veracrypt/mp4 file
    This challenge relies on two core steganographic techniques, which are
    combined with more common ones -- the latter being described in the next
    section. Both of these core techniques use the way some file formats are
    specified to embed additional content within them without making the
    original file unreadable.\\

    The first technique is called \textit{Angecryption}\footnote{See \href{https://github.com/indrora/corkami/blob/master/src/angecryption/slides/AngeCryption.pdf}{github.com/indrora/corkami/blob/master/src/angecryption/slides/AngeCryption.pdf} for more details}.
    Its concept is simple and yet powerful: it consists in using AES-128 CBC to
    encrypt a file in one format to a valid file in another format, by choosing
    the IV so that the first block of the target (second) format - considered
    as a ciphertext block - decrypts into a plaintext block corresponding to
    the beginning of a file of the first format. The original algorithm was
    written in Python 2 and covered file formats such as PNG and PDF; in the
    framework of this project, we ported it to Python 3 and added support for
    raw MP3 files as well as for MP3 files with metadata.

    The second core technique used to build this challenge consists in embedding
    a VeraCrypt volume in a MP4 file, so that both the encrypted volume and the
    original video are still readable. More specifically, VeraCrypt volumes can
    be created in way that allow plausible deniablility: you can create a
    first, outer volume, and create a second, hidden volume within the first
    one. You then put content in the outer volume that can seem interesting to a
    third party but contains nothing critical, and put the actual data you
    want to conceal in the inner volume. Thus, in a situation where the
    existence of a VeraCrypt volume is discovered by a adversarial third
    party who asks you to mount and decrypt the volume, you can provide the
    passphrase of the outer volume, displaying only the data you do not really
    care about without arousing the suspicion of the adversarial third party,
    who cannot prove there is more data than you are claiming there is.\\
    \indent The structure of a VeraCrypt container makes it possible to write
    the headers of a MP4 file on the part of the file that originally contains
    the headers of the outer volume. With some more tweaks, one can get a file
    that is a valid MP4 video file and also contains the hidden Veracrypt volume.
    The outer volume is lost in that hybridization process.\\
    \indent Previous works had been carried out on MP4-TrueCrypt hybridization.
    In the context of this project, we implemented a solution that allows
    MP4-VeraCrypt hybridization.\\

    In the next section, we will describe more specifically what steganographic
    techniques are at stake in each step of the challenge, and how the different
    step are connected to each other.


\section*{Matryoshkryption -- Detailed description}
    % structure of the chall
    % techniques used at each step
    The challenge consists of four steps and is divided in two parts, each
    corresponding to two steps of the whole challenge. At each step, the
    challenger gets a file of a new type, and has to find cryptographic
    material concealed in it to proceed to the next step via Angecryption.\\

    The challenger initially gets a \texttt{.tar.gz} archive, which once
    decompressed reveals a PNG file. The steganographic technique used in that
    step to hide cryptographic material as well as an additional element is
    that of the LSB (\textbf{L}east \textbf{S}ignificant \textbf{B}it), where
    the least significant bit of the bytes that make each component (red,
    green, blue) of a pixel is altered to embed part of the data to conceal. In
    our case, the symmetric key is hidden in the plan made of the LSBs of the
    green component, the IV in that of the blue component. Both key and IV are
    hidden through "visual" LSB: one has to look at the black and white picture
    resulting of the extraction of the LSB plan to see the hidden message.
    Besides, the LSB plan of the red component holds an dictionary that will be
    needed by the challenger later on.\\ \indent The challenger must then write
    a script that "de-angecrypts" the file using the found key/IV pair. The
    process of decryption reveals a PDF file.\\

    The second step consists in a PDF document, which displays a music score.
    The dictionary found in the previous step must be used to decode the notes
    into a text message, which provides the challenger with a new key/IV pair,
    and with an intermediate flag. Additionally, the copyright line of the
    music score holds a pasphrase that will be of use later on. With the new
    key/IV pair, the challenger can proceed to the next step.\\

    After decrypting the PDF document, one gets an audio file in a format
    which is non-standard but close to the MP3 file format. The cryptographic
    material for this step has been embedded in two different ways: the
    symmetric key is simply morse-encoded, and the IV is hidden in the
    spectrogram of the file -- basically, a picture showing the IV is used to
    generate an audio file whose spectrography has the aspect of that picture.
    The audio file does not hold any additional content. Once again,
    decrypting the current file lets the challenger proceed to the next
    one, which appears to be a MP4 video file.\\

    The last step consists of what first seems to be a mere MP4 file.
    However, the file is actually the result of the hybridization of a MP4 file
    and a VeraCrypt volume, as described in the previous section. The passphrase
    the challenger got from the PDF document leads to that information, and can
    then be used to decrypt the volume and mount it. The volume contains a
    single file, whose content is the final flag of the challenge.\\



    % overview
    A graph showing the different steps and their nesting can be found in Fig.\ref{fig:overview}.

\newgeometry{textwidth=19cm}
\begin{figure}[H]
\tikzset{stepnode/.style={draw,rectangle,rounded corners=3pt,minimum height=1cm,minimum width=2cm,fill=gray!30}}
\tikzset{flagnode/.style={draw,rectangle,rounded corners=3pt,minimum height=1cm,minimum width=2cm,align=center,fill=green!30}}
\tikzset{stepinfo/.style={auto,rectangle,rounded corners=3pt,minimum height=1cm,minimum width=2cm,text width=#1},
        stepinfo/.default={2cm}}
\tikzset{hiddennode/.style={}}

\tikzset{nodearrow/.style={thick, ->, >=stealth}}
\tikzset{infoarrow/.style={thick, ->, >=stealth}}
\tikzset{flagarrow/.style={very thick, ->, >=stealth, color=green}}

\begin{center}

    \begin{tikzpicture}[auto, scale=\textwidth/21cm, transform shape]
        \node[stepnode] (PNG){png};
        \node[stepnode,right= 1.5cm of PNG] (PDF){pdf};
        \node[stepnode,right= 1.5cm of PDF] (MP3){mp3};
        \node[stepnode,right= 1.5cm of MP3] (MP4VC){mp4 / VeraCrypt volume};
        \node[flagnode,right= 1.5cm of MP4VC] (F){\textbf{Flag}};
        \node[flagnode,text width = 2.8cm,below right= 3.5cm and 0.5cm of PDF] (IF){\textbf{Intermediate Flag}};
        \node[stepinfo, below= 4cm of PNG, text width = 5cm] (PNGINFO){key: visual LSB (G)
~~~~~            iv: ~~visual LSB (B)};
        \node[stepinfo, below right= 5cm and -3.7cm of PNG, color=red, text width = 3.2cm] (PDFALPHA){dict.: LSB (R)};
        \node[stepinfo, below right= 5.2cm and -2.2cm of PDF, color=red, text width = 3.5 cm] (VCPASS){passphrase for VC volume: footer};

        \node[stepinfo, below= 4cm of PDF, text width= 2.1cm] (PDFINFO){key + iv:
            music score};
        \node[stepinfo, below right= 1.5cm and -1.5cm of MP3, text width= 2.1cm] (MP3INFO){key: morse
            iv: spectro};
        \node[hiddennode, below= 1.5cm of MP3, text width= 2.1cm] (MP3H){};
        %\node[stepinfo, below right= 4cm and -3cm  of MP4, text width= 4.8cm] (MP4INFO){passphrase: tagline ~~~~~~~ of the music score};
        %\node[hiddennode, below right= 3.8cm and -1cm of MP4, text width= 4.5cm] (MP4H){};
        %\node[stepinfo, below right= 4cm and -3.3cm of VC, text width= 4cm] (VCINFO){\texttt{flag} file containing the flag};
        %\node[hiddennode, below right= 3.8cm and -1.8cm of VC, text width= 4.5cm] (VCH){};


        \draw[nodearrow] (PNG)--(PDF);
        \draw[nodearrow] (PDF)--(MP3);
        \draw[nodearrow] (MP3)--(MP4VC);
%        \draw[nodearrow] (MP4)--(VC);
        \draw[flagarrow] (MP4VC)--(F);
        \draw[flagarrow, dashed] (PDF.-45) to[out=-75, in=110](IF.north);

        \draw[infoarrow] (PNG)--(PNGINFO);
        \draw[infoarrow, dashed, color=red] (PDFALPHA.east) to[out=15, in=-125](PDF.225);
        \draw[infoarrow] (PDF)--(PDFINFO);
        \draw[infoarrow, dashed, color=red] (VCPASS.east) to[out=0, in=-105](MP4VC.225);
        \draw[infoarrow] (MP3)--(MP3H);
%        \draw[infoarrow] (MP4)--(MP4H.west);
%        \draw[infoarrow] (VC)--(VCH.west);

    \end{tikzpicture}
    \caption{Challenge Overview}
    \label{fig:overview}
\end{center}
\end{figure}
\restoregeometry


\end{document}
